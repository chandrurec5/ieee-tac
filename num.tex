%!TEX root =  autocontgrlp.tex
\section{Numerical Illustration}
In this section, we show via an example in the domain of controlled queues the consequences of \cref{conetheorm},
which bounded the error when the constraints are chosen based on selecting a set of representative states
% (the \emph{cone} condition) derived based on the geometric intuition 
 (further preliminary experimental results have been reported in \cite{aaaipaper}).

\textbf{Model:} We run the experiments in the context of a queuing model similar to the one in Section~5.2 of \cite{ALP}. We consider a single queue with arrivals and departures. The state of the system is the queue length with the state space given by $\S=\{0,\ldots,S-1\}$, where $S-1$ is the buffer size of the queue. The action set $\A=\{1,\ldots,A\}$ is related to the service rates. We let $s_t$ denote the state at time $t$. The state at time $t+1$ when action $a_t \in \A$ is chosen is given by $s_{t+1}= s_{t}+1$ with probability $p$, $s_{t+1}= s_{t}-1$ with probability $q(a_t)$ and $s_{t+1}= s_t$, with probability $(1-p-q(a_t))$. For states $s_t=0$ and $s_t=S-1$, the system dynamics is given by $s_{t+1}= s_{t}+1$ with probability $p$ when $s_t=0$ and $s_{t+1}=s_t-1$ with probability $q(a_t)$ when $s_t=S-1$. The service rates satisfy $0<q(1)\leq \ldots\leq q(A)<1$ with $q(A)>p$ so as to ensure `stabilizability' of the queue.
The reward associated with  action $a \in \A$ and state $s\in \S$ is given by $g_a(s)=-(s/N+q(a)^3)$ (the idea here is to penalize higher queue lengths and higher service rates).
\textbf{Parameter Settings:} We ran our experiments for $S=1000$, $A=4$ with $q(1)=0.2$, $q(2)=0.4$, $q(3)=0.6$, $q(4)=0.8$, $p=0.4$ and $\alpha=1-\frac{1}{S}$.
The moderate size of $S=1000$ enabled us to compute the exact value of $J^*$ (the most expensive part of the computation). 
Despite that the conic span condition cannot be met with them, as a way of testing the limits of the theory,
we made use of polynomial features in $\Phi$ (i.e., $1,s,\ldots,s^{k-1}$) 
since they are known to work reasonably well for this domain \cite{ALP}.
We chose $k=4$. % (i.e., we used $1, s,s^2$ and $s^3$ as basis vectors).

\textbf{Experimental Methodology:} We compare two different sampling strategies $(i)$ based on the \emph{cone} conditions, and $(ii)$ based on constraint sampling. The two strategies are compared via  \emph{lookahead} policies, wherein, the action at state $s$ is obtained by computing the approximate value functions of the next states. The details are as follows:
\emph{Case (i)}: Except for the corner states i.e., $s=0$ and $s=999$, each state $0<s<S-1$ has two next states namely $s'=s-1$ and $s'=s+1$. We formulate two separate LRALPs (or just one LRALP for $s=0$ and $s=S-1$) for next states. When formulating the LRALP for state $s'$, we let $c=e_{s'}$ and choose the constraint corresponding to state $s'$ to ensure the cone condition to be met for LRALP. We choose $5$ more constraints corresponding to states $1,200,400,600,800,999$ (uniformly spaced across the state space) and compute $\hat{J}_{e_{s'}}$. The lookahead policy is formulated as
$u_{LRA}(s)=\argmin_{a\in A} g_a(s)+\sum_{s'\in S}p_a(s,s')\hat{J}_{e_{s'}}(s')$.

\emph{Case (ii)}: In a manner similar to \emph{Case (i)}, here too we formulate two separate LRALPs for next states. When formulating the LRALP for state $s'$, we first let $c_{s'}(s)=(1-\alpha)(\alpha)^{|s'-s|}$ and normalize $c$ and sample $m=6$ constraints randomly using $c$ and compute $\hat{J}_{c_s'}$. Since the optimal sampling distribution is unknown, we sample from $c$ which assigns higher weights to nearby states.
The lookahead policy is formulated as 
$u_{CS}(s)=\argmin_{a\in A} g_a(s)+\sum_{s'\in S}p_a(s,s')\hat{J}_{c_{s'}}(s')$.

The results are shown in the figures below:
\FloatBarrier
\begin{figure}[htp]
\begin{minipage}{0.5\textwidth}
\resizebox{1.0\textwidth}{!}{
\begin{tabular}{cc}
\begin{tikzpicture}[scale=1,font=\Large,]
    \begin{axis}[]
    \addplot[smooth,very thick, mark=diamond, each nth point=100] plot file {../lralp_exp/V_LP};
    \addplot[dashed,very thick, mark=+, each nth point=100] plot file {../lralp_exp/V_pol_CS};
    \addplot[dotted,very thick,mark=square, each nth point=100] plot file {../lralp_exp/V_pol_cone};
    \addlegendentry{$J^*$};
    \addlegendentry{$J_{CS}$};
    \addlegendentry{$J_{LRA}$};

\end{axis}
\end{tikzpicture}

&
\begin{tikzpicture}[scale=1,font=\Large]
    \begin{axis}[]
\addplot[very thick] plot file {../lralp_exp/policy};
\addplot[dashed,very thick] plot file {../lralp_exp/policy_CS};
\addplot[dotted,very thick] plot file {../lralp_exp/policy_cone};
\addlegendentry{$u^*$};
\addlegendentry{$u_{CS}$};
\addlegendentry{$u_{LRA}$};

    \end{axis}
    \end{tikzpicture}

\end{tabular}
}
\end{minipage}
\end{figure}

The right-hand-side figure shows the policies computed, 
while the left-hand-side figure shows their value functions. 
Since constraint sampling (CS) produces randomized results,
we repeated the simulations 10 times. 
The results in all cases were quite similar except for one ``bad case'' for CS , hence we show the plot for a typical run.
As can be seen from the figure, choosing the constraints to (approximately) satisfy the constraint of the theoretical results
reliably produces better results: In fact, the value functions $J^*$ and $J_{LRA}$ are mostly on the top of each other.
While these results are only meant for illustration,
we expect that in larger domains it becomes even more important to select the constraints appropriately.
However, the study of this is left for future work.

